\documentclass[10pt]{article}

%% Various useful packages and commands from different sources

\usepackage[applemac]{inputenc}
\usepackage[english]{babel}
\usepackage[T1]{fontenc}
\usepackage{cite, url,color} % Citation numbers being automatically sorted and properly "compressed/ranged".
%\usepackage{pgfplots}
\usepackage{graphics,amsfonts}
\usepackage[pdftex]{graphicx}
\usepackage[cmex10]{amsmath}
% Also, note that the amsmath package sets \interdisplaylinepenalty to 10000
% thus preventing page breaks from occurring within multiline equations. Use:
 \interdisplaylinepenalty=2500
% after loading amsmath to restore such page breaks as IEEEtran.cls normally does.

% Compact lists
\usepackage{enumitem}
\usepackage{booktabs}
\usepackage{fancyvrb}

\usepackage{listings} % for Matlab code
\definecolor{commenti}{rgb}{0.13,0.55,0.13}
\definecolor{stringhe}{rgb}{0.63,0.125,0.94}
\lstloadlanguages{Matlab}
\lstset{% general command to set parameter(s)
framexleftmargin=0mm,
frame=single,
keywordstyle = \color{blue},% blue keywords
identifierstyle =, % nothing happens
commentstyle = \color{commenti}, % comments
stringstyle = \ttfamily \color{stringhe}, % typewriter type for strings
showstringspaces = false, % no special string spaces
emph = {for, if, then, else, end},
emphstyle = \color{blue},
firstnumber = 1,
numbers =right, %  show number_line
numberstyle = \tiny, % style of number_line
stepnumber = 5, % one number_line after stepnumber
numbersep = 5pt,
language = {Matlab},
extendedchars = true,
breaklines = true,
breakautoindent = true,
breakindent = 30pt,
basicstyle=\footnotesize\ttfamily
}

\usepackage{array}
% http://www.ctan.org/tex-archive/macros/latex/required/tools/
\usepackage{mdwmath}
\usepackage{mdwtab}
%mdwtab.sty	-- A complete ground-up rewrite of LaTeX's `tabular' and  `array' environments.  Has lots of advantages over
%		   the standard version, and over the version in `array.sty'.
% *** SUBFIGURE PACKAGES ***
\usepackage[tight,footnotesize]{subfigure}
\usepackage[top=2.2cm, bottom=2.2cm, right=1.7cm,left=1.7cm]{geometry}
\usepackage{indentfirst}


%\setlength\parindent{0pt}
\linespread{1}

\usepackage{mathtools}
\DeclarePairedDelimiter{\ceil}{\lceil}{\rceil}
\DeclarePairedDelimiter{\floor}{\lfloor}{\rfloor}
\DeclareMathOperator*{\argmax}{arg\,max}
\newcommand{\M} {\mathtt{M}}
\newcommand{\dB} {\mathrm{dB}}
\newcommand{\tr} {\mathrm{tr}}



\graphicspath{ {figures/} }

% equations are numbered section by section
%\numberwithin{equation}{section}


\begin{document}
\title{Digital Transmission - Homework 3}
\author{Andrea Dittadi, Davide Magrin, Michele Polese}

\maketitle

%%%%%%%%%%%%%%%%%%%%%%%%%%%%%%%%%%%%%%%%%
%%%%%%%%%%%%%% PROBLEM 1 %%%%%%%%%%%%%%%%
%%%%%%%%%%%%%%%%%%%%%%%%%%%%%%%%%%%%%%%%%

\section*{Problem 1}

% Spiegare cosa mandiamo e come lo generiamo
In this problem the channel $q(n\frac{T}{4})$ is given at $\frac{T}{4}$. The goal is to estimate at the receiver the timing phase $t_0$ and the equivalent channel impulse response in $T$ with LS method.
The setting of the system is as in Fig.~\ref{fig:channel_T4}. In order to perform LS estimation the first $L + N_{seq} = 15 + 10$ symbols of the sequence $a_k$ are binary symbols out of a QPSK constellation with $\sigma_a^2 = 2$. 
In particular they are $1+j$ and $-1-j$ and map the $+1$ and $-1$ values of a ML sequence of length $L = 15$, partially repeated by $N_{seq}=10$. Then QPSK data follows. 
The conversion between a bit stream and QPSK is carried out by the \texttt{bitmap} function. %% TODO some more on this?? 

\begin{figure}
	\centering
	\includegraphics[width = 0.8\textwidth]{channel_T4}
	\caption{System setting for problem 1}
	\label{fig:channel_T4}
\end{figure}

%%TODO Does Eq go in dB???? 

% Spiegare che channel_output funziona da polifase
The noisy output $r_n$ of the channel is computed with a polyphase version of the channel impulse response $q(n\frac{T}{4})$. Each branch is $q^{i}(mT) = q(mT + i)$ with $i \in [0, 3]$ , and since it is not time-varying it is possible to directly compute the convolution of each branch and the $a_k$ sequence. At the output of each convolution white noise $w_n$ is added, with variance $\sigma_w^2 = \frac{\sigma_a^2 E_q}{4 \Gamma} = -19.2411$ dB, where $\sigma_a^2 = 2$, $E_q = 2.3819$ and $\Gamma = 20$ dB is the given SNR at receiver input. Then a parallel to series converter generates the output at time $\frac{T}{4}$.

% !!! Spiegare come stimiamo t0, riportare l'equazione (Formula 7.269, channel_estimator)
The first operation which the receiver does is estimating $t_0$ as optimal timing phase in $\frac{T}{4}$. In particular, given the stream of symbols $r_n$, and the knowledge of the ML sequence $a_l, l\in [0, L-1]$, the receiver computes

\begin{equation}
	m_{opt} = \argmax_m | x_{ra}(m\frac{T}{4}) | = \argmax_{m} | \frac{1}{L} \sum_{l = 0}^{L - 1} x(lT + m\frac{T}{4})a^* (l) |
\end{equation}
for $ 0 < m < 40$ and finds that
\begin{equation}
	m_{opt} = 10
\end{equation}
\begin{equation}
	t_{0} = m_{opt}\frac{T}{4} = 10 \frac{T}{4}
\end{equation}

% Spiegare differenza tra m_opt (in T/4, che è t0), init_offs (in T/4) e delay (in T)
% Is it really useful? Michele


% Funzionale per la scelta di N1 e N2 
% 1 - Considerations sulla modification della formulation per il metodo ls visto che la sequenza ha simboli che distano 2sqrt(2)
% (2 - Considerazioni su scelta di N1 legata al funzionale che sta sempre lì e cambia proprio poco - anche in funzione dell'esercizio 2)

% !!! Riportare l'equazione per la stima di {h_i_hat}

% !!! Riportare valori di {h_i_hat} in una tabella, ampiezza e fase

% !!! Plottare {h_i_hat} per i = -N1, ..., N2

% !!! Determinare la stima della varianza del rumore, sigma_w_hat_squared, compararla con sigma_w_squared in dB.
The noise that the channel introduces is $\sigma_w^2 = -19.2411$ dB, while the estimate $\hat{\sigma}_w^2 = something = take value from report$ is lower. Note that this is the value given by a single estimate and not an expectation. However, this shows that a sequence of the given length $L = 15$ is too short to correctly estimate the real noise of the channel, since it averages on a number of noise samples which is close to the length of the impulse response we are trying to estimate. Indeed, by using a longer ML sequence (say 127) the estimate obtained is much closer to the real value also in each single realization. 
% !!! Comparare la stima di Lambda_n_hat con il valore "vero" di Lambda_n_hat, fare le considerazioni su LS per cui il Lambda_n teorico raddoppia quando si usano i simboli ±1±j al posto di ±1.

\clearpage

%%%%%%%%%%%%%%%%%%%%%%%%%%%%%%%%%%%%%%%%%
%%%%%%%%%%%%%% PROBLEM 2 %%%%%%%%%%%%%%%%
%%%%%%%%%%%%%%%%%%%%%%%%%%%%%%%%%%%%%%%%%

\section*{Problem 2}

% Breve descrizione del sistema in T

% Se non fatto nel problema 1 riportare le considerazioni su N1

% Considerazioni sull'implementazione di LE e DFE, riportare i grafici di Jmin (qualche grafico), considerazioni su M1, M2, D, riportare i grafici di \hat{h}, c, psi, b per \Gamma = 10 dB

% Considerazioni su Viterbi, Kd, grafico della distribuzione

% Considerazioni su FBA, implementazioni ecc

% Grafici del BER, sia per canale stimato che per canale vero, riportando il numero di bit per le simulazioni



\begin{thebibliography}{10}

\bibitem{bc}
Benvenuto, Cherubini, Algorithms for Communications Systems and their Applications, Wiley, 2004

\end{thebibliography}

\end{document}
