\documentclass[10pt]{article}

%% Various useful packages and commands from different sources

\usepackage[applemac]{inputenc}
\usepackage[english]{babel}
\usepackage[T1]{fontenc}
\usepackage{cite, url,color} % Citation numbers being automatically sorted and properly "compressed/ranged".
%\usepackage{pgfplots}
\usepackage{graphics,amsfonts}
\usepackage[pdftex]{graphicx}
\usepackage[cmex10]{amsmath}
\usepackage{bm}
% Also, note that the amsmath package sets \interdisplaylinepenalty to 10000
% thus preventing page breaks from occurring within multiline equations. Use:
 \interdisplaylinepenalty=2500
% after loading amsmath to restore such page breaks as IEEEtran.cls normally does.

% Compact lists
\usepackage{enumitem}
\usepackage{booktabs}
\usepackage{fancyvrb}

\usepackage{listings} % for Matlab code
\definecolor{commenti}{rgb}{0.13,0.55,0.13}
\definecolor{stringhe}{rgb}{0.63,0.125,0.94}
\lstloadlanguages{Matlab}
\lstset{% general command to set parameter(s)
framexleftmargin=0mm,
frame=single,
keywordstyle = \color{blue},% blue keywords
identifierstyle =, % nothing happens
commentstyle = \color{commenti}, % comments
stringstyle = \ttfamily \color{stringhe}, % typewriter type for strings
showstringspaces = false, % no special string spaces
emph = {for, if, then, else, end},
emphstyle = \color{blue},
firstnumber = 1,
numbers =right, %  show number_line
numberstyle = \tiny, % style of number_line
stepnumber = 5, % one number_line after stepnumber
numbersep = 5pt,
language = {Matlab},
extendedchars = true,
breaklines = true,
breakautoindent = true,
breakindent = 30pt,
basicstyle=\footnotesize\ttfamily
}

\usepackage{array}
% http://www.ctan.org/tex-archive/macros/latex/required/tools/
\usepackage{mdwmath}
\usepackage{mdwtab}
%mdwtab.sty	-- A complete ground-up rewrite of LaTeX's `tabular' and  `array' environments.  Has lots of advantages over
%		   the standard version, and over the version in `array.sty'.
% *** SUBFIGURE PACKAGES ***
\usepackage[tight,footnotesize]{subfigure}
\usepackage[top=2.2cm, bottom=2.2cm, right=1.7cm,left=1.7cm]{geometry}
\usepackage{indentfirst}


%\setlength\parindent{0pt}
\linespread{1}

\usepackage{mathtools}
\DeclarePairedDelimiter{\ceil}{\lceil}{\rceil}
\DeclarePairedDelimiter{\floor}{\lfloor}{\rfloor}
\DeclareMathOperator*{\argmax}{arg\,max}
\newcommand{\M} {\mathtt{M}}
\newcommand{\dB} {\mathrm{dB}}
\newcommand{\tr} {\mathrm{tr}}
\newcommand{\lmod}[1] {_{\,\mathrm{mod}\,#1}}



\graphicspath{ {figures/} }

% equations are numbered section by section
%\numberwithin{equation}{section}


\begin{document}
\title{Digital Transmission - Homework 4}
\author{Andrea Dittadi, Davide Magrin, Michele Polese}

\maketitle

%%%%%%%%%%%%%%%%%%%%%%%%%%%%%%%%%%%%%%%%%
%%%%%%%%%%%%%% PROBLEM 1 %%%%%%%%%%%%%%%%
%%%%%%%%%%%%%%%%%%%%%%%%%%%%%%%%%%%%%%%%%

\section*{Problem 1}



%%%%%%%%%%%%%%%%%%%%%%%%%%%%%%%%%%%%%%%%%
%%%%%%%%%%%%%% PROBLEM 2 %%%%%%%%%%%%%%%%
%%%%%%%%%%%%%%%%%%%%%%%%%%%%%%%%%%%%%%%%%

\section*{Problem 2}

%%%%%%%%%%%%%%%%%%%%%%%%%%%%%%%%%%%%%%%%%
%%%%%%%%%%%%%% PROBLEM 2 %%%%%%%%%%%%%%%%
%%%%%%%%%%%%%%%%%%%%%%%%%%%%%%%%%%%%%%%%%
\section*{Problem 3}
As usual the SNR is defined as $\Gamma = \frac{\sigma_s^2 E_h}{\sigma_w^2}$. With an OFDM system, however, the meaning of $\sigma_s^2$ changes. In particular if the transmitter is implemented with an IFFT, which is
\begin{equation}
	A_k[\ell] = \frac{1}{\mathcal{M}} \sum_{i = 0}^{\mathcal{M} - 1} W_{\mathcal{M}}^{-i\ell} a_k[i], \quad \ell = 0, 1, \dots, \mathcal{M}-1
\end{equation}
as in the MATLAB implementation, followed by the polyphase compoment of a prototype filter, which in our case is a $\delta_n$ filter, with a gain $\mathcal{M}$ as specified in \cite{bc}, then we have 
% see formula 9.35, 9.38, 9.39
\begin{equation}
	\sigma_s^2 = \mathcal{M} \sigma_a^2
\end{equation}
since data is IID. 

Therefore, if the energy of the data symbols is considered, then
\begin{equation}
	\Gamma = \frac{\mathcal{M} \sigma_a^2 E_h}{\sigma_w^2}
\end{equation}
must be the SNR considered when computing the noise level introduced by the channel.

\begin{thebibliography}{10}

\bibitem{bc}
Benvenuto, Cherubini, Algorithms for Communications Systems and their Applications, Wiley, 2004

\end{thebibliography}

\end{document}
